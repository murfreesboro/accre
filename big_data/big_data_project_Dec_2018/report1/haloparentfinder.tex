%!TeX encoding = UTF-8
%!TeX program = xelatex
\documentclass[notheorems, aspectratio=54]{beamer}
% aspectratio: 1610, 149, 54, 43(default), 32

\usepackage{latexsym}
\usepackage{amsmath,amssymb}
\usepackage{mathtools}
\usepackage{color,xcolor}
\usepackage{graphicx}
\usepackage{algorithm}
\usepackage{amsthm}
\usepackage{lmodern} % 解决 font warning
% \usepackage[UTF8]{ctex}
\usepackage{animate} % insert gif

\usepackage{lipsum} % To generate test text 
\usepackage{ulem} % 下划线,波浪线

\usepackage{listings} % display code on slides; don't forget [fragile] option after \begin{frame}
\usepackage{verbatim}
\makeatletter
\def\verbatim@font{\tiny\ttfamily}
\makeatother

% ----------------------------------------------
% tikx
\usepackage{framed}
\usepackage{tikz}
\usepackage{pgf}
\usetikzlibrary{calc,trees,positioning,arrows,chains,shapes.geometric,%
    decorations.pathreplacing,decorations.pathmorphing,shapes,%
    matrix,shapes.symbols}
\pgfmathsetseed{1} % To have predictable results
% Define a background layer, in which the parchment shape is drawn
\pgfdeclarelayer{background}
\pgfsetlayers{background,main}

\definecolor{AmethystPurple}{HTML}{AEAEDF}
% define styles for the normal border and the torn border
\tikzset{
  normal border/.style={AmethystPurple, decorate, 
     decoration={random steps, segment length=2.5cm, amplitude=.7mm}},
  torn border/.style={AmethystPurple, decorate, 
     decoration={random steps, segment length=.5cm, amplitude=1.7mm}}}

% Macro to draw the shape behind the text, when it fits completly in the
% page
\def\parchmentframe#1{
\tikz{
  \node[inner sep=1.5em] (A) {#1};  % Draw the text of the node
  \begin{pgfonlayer}{background}  % Draw the shape behind
  \fill[normal border] 
        (A.south east) -- (A.south west) -- 
        (A.north west) -- (A.north east) -- cycle;
  \end{pgfonlayer}}}

% Macro to draw the shape, when the text will continue in next page
\def\parchmentframetop#1{
\tikz{
  \node[inner sep=2em] (A) {#1};    % Draw the text of the node
  \begin{pgfonlayer}{background}    
  \fill[normal border]              % Draw the ``complete shape'' behind
        (A.south east) -- (A.south west) -- 
        (A.north west) -- (A.north east) -- cycle;
  \fill[torn border]                % Add the torn lower border
        ($(A.south east)-(0,.2)$) -- ($(A.south west)-(0,.2)$) -- 
        ($(A.south west)+(0,.2)$) -- ($(A.south east)+(0,.2)$) -- cycle;
  \end{pgfonlayer}}}

% Macro to draw the shape, when the text continues from previous page
\def\parchmentframebottom#1{
\tikz{
  \node[inner sep=2em] (A) {#1};   % Draw the text of the node
  \begin{pgfonlayer}{background}   
  \fill[normal border]             % Draw the ``complete shape'' behind
        (A.south east) -- (A.south west) -- 
        (A.north west) -- (A.north east) -- cycle;
  \fill[torn border]               % Add the torn upper border
        ($(A.north east)-(0,.2)$) -- ($(A.north west)-(0,.2)$) -- 
        ($(A.north west)+(0,.2)$) -- ($(A.north east)+(0,.2)$) -- cycle;
  \end{pgfonlayer}}}

% Macro to draw the shape, when both the text continues from previous page
% and it will continue in next page
\def\parchmentframemiddle#1{
\tikz{
  \node[inner sep=2em] (A) {#1};   % Draw the text of the node
  \begin{pgfonlayer}{background}   
  \fill[normal border]             % Draw the ``complete shape'' behind
        (A.south east) -- (A.south west) -- 
        (A.north west) -- (A.north east) -- cycle;
  \fill[torn border]               % Add the torn lower border
        ($(A.south east)-(0,.2)$) -- ($(A.south west)-(0,.2)$) -- 
        ($(A.south west)+(0,.2)$) -- ($(A.south east)+(0,.2)$) -- cycle;
  \fill[torn border]               % Add the torn upper border
        ($(A.north east)-(0,.2)$) -- ($(A.north west)-(0,.2)$) -- 
        ($(A.north west)+(0,.2)$) -- ($(A.north east)+(0,.2)$) -- cycle;
  \end{pgfonlayer}}}

% Define the environment which puts the frame
% In this case, the environment also accepts an argument with an optional
% title (which defaults to ``Example'', which is typeset in a box overlaid
% on the top border
\newenvironment{parchment}[1][Example]{%
  \def\FrameCommand{\parchmentframe}%
  \def\FirstFrameCommand{\parchmentframetop}%
  \def\LastFrameCommand{\parchmentframebottom}%
  \def\MidFrameCommand{\parchmentframemiddle}%
  \vskip\baselineskip
  \MakeFramed {\FrameRestore}
  \noindent\tikz\node[inner sep=1ex, draw=black!20,fill=AmethystPurple, 
          anchor=west, overlay] at (0em, 1em) {\sffamily#1};\par}%
{\endMakeFramed}

% ----------------------------------------------

\mode<presentation>{
    \usetheme{Berkeley}
    % Boadilla CambridgeUS
    % default Antibes Berlin Copenhagen
    % Madrid Montpelier Ilmenau Malmoe
    % Berkeley Singapore Warsaw
    \usecolortheme{dolphin}
    % beetle, beaver, orchid, whale, dolphin
    \useoutertheme{infolines}
    % infolines miniframes shadow sidebar smoothbars smoothtree split tree
    \useinnertheme{circles}
    % circles, rectanges, rounded, inmargin
}
% 设置 block 颜色
\setbeamercolor{block title}{bg=AmethystPurple,fg=white}

\newcommand{\reditem}[1]{\setbeamercolor{item}{fg=red}\item #1}

% 缩放公式大小
\newcommand*{\Scale}[2][4]{\scalebox{#1}{\ensuremath{#2}}}

% 解决 font warning
\renewcommand\textbullet{\ensuremath{\bullet}}

% verbatim


% ---------------------------------------------------------------------
% flow chart
\tikzset{
    >=stealth',
    punktchain/.style={
        rectangle, 
        rounded corners, 
        % fill=black!10,
        draw=white, very thick,
        text width=6em,
        minimum height=2em, 
        text centered, 
        on chain
    },
    largepunktchain/.style={
        rectangle,
        rounded corners,
        draw=white, very thick,
        text width=10em,
        minimum height=2em,
        on chain
    },
    line/.style={draw, thick, <-},
    element/.style={
        tape,
        top color=white,
        bottom color=blue!50!black!60!,
        minimum width=6em,
        draw=blue!40!black!90, very thick,
        text width=6em, 
        minimum height=2em, 
        text centered, 
        on chain
    },
    every join/.style={->, thick,shorten >=1pt},
    decoration={brace},
    tuborg/.style={decorate},
    tubnode/.style={midway, right=2pt},
    font={\fontsize{10pt}{12}\selectfont},
}
% ---------------------------------------------------------------------

% code setting
\lstset{
    language=C++,
    basicstyle=\ttfamily\footnotesize,
    keywordstyle=\color{red},
    breaklines=true,
    xleftmargin=2em,
    numbers=left,
    numberstyle=\color[RGB]{222,155,81},
    frame=leftline,
    tabsize=4,
    breakatwhitespace=false,
    showspaces=false,               
    showstringspaces=false,
    showtabs=false,
    morekeywords={Str, Num, List},
}

% ---------------------------------------------------------------------

%% preamble
\title[Halo Parent Finder]{HaloParentFinder}
\subtitle{Big Data Project Transformation}
\author{Fenglai Liu}
\institute[ACCRE]{fenglai@accre.vanderbilt.edu}

% -------------------------------------------------------------

\begin{document}

%% title frame
\begin{frame}
    \titlepage
\end{frame}

% -------------------------------------------------------------
\section{Introduction}
\begin{frame}
%    \frametitle{}

This program(HaloParentFinder) is about 2500 lines of code (including comments). 
This is the first program, there are other two programs following the HaloParentFinder:
\begin{itemize}
 \item HaloParentFinder;
 \item OrphanFixer;
 \item MergerTree
\end{itemize}

\end{frame}

% -------------------------------------------------------------
\begin{frame}
%    \frametitle{}

General procedure for the HaloParentFinder:
\begin{itemize}
 \item Load in parameters. The global parameters are stored in a global struct PARAMS;
 \item Calculate the number of groups - group0 and group1;
 \item Allocate space for group0 and group1 and loading in the group data; 
 \item Finds the hierarchy level (this is for a single snapshot);
 \item Find fof parents for groups in group0 based on group1;
 \item Find all parents for groups in group0 based on group1;
 \item If the halo does not have parent, loop over the remaining snapshot data and repeat the previous steps;
 \item Perform additional checks to make sure the data is consistent.
\end{itemize}


\end{frame}


% -------------------------------------------------------------
\section{Components}
\subsection{Parameters Reading}
\begin{frame}
%    \frametitle{}

The codes for reading the parameters are in the read\_param.c and read\_param.h. The parameters defines
the location of the groups data(input), and output files directory; the limits of snapshots 
and other debugging options (one data file is a snapshot). The developer use a global variable PARAMS to store 
the parameters information. 

\end{frame}



% -------------------------------------------------------------
\subsection{Halo Groups}
\begin{frame}
%    \frametitle{}

``group'' is the central idea for this program - the structure of group is defined in io.h. Group is a collection of particles(Halo). One data file (snapshot)
may containing many groups.

\begin{block}{reading in the group data}
 \begin{itemize}
  \item first line is the the number of groups data;
  \item for each group data first line is the number of particles and the halo ID;
  \item the following lines are the particles data, like  x, y, z, vx, vy, vz etc. These data in group struct are all allocated based on the number of particles.
 \end{itemize}
\end{block}


In the code the group\_data pointer is an array of groups. The group data is initialized in the  the loadgroups function.

\end{frame}

% -------------------------------------------------------------
\begin{frame}
%    \frametitle{}

\begin{block}{Core reason for the memory insufficiency}
In the group array the majority of the data is for particles, like particles $x$, $y$
and $z$. For the large snapshot who contains $10^5$ groups, and each group contains around $10^5$ particles; it's clear to see when loading in all of particles data from the snapshot the particles data uses most of the memory.
\end{block}

\begin{alertblock}{How to improve}
\begin{itemize}
 \item the bottleneck step (parent finding process) seems only using particles ID data, so we do not need to carry all of particles data around;
 \item we can load in part/whole snapshot (this is also required for big data transformation);
 \item use high level data structure from the language - STL vector in C++, vector class from Java etc. so that we do not need to allocate/free group data explicitly. 
\end{itemize}

 
\end{alertblock}



\end{frame}

% -------------------------------------------------------------
\subsection{Functions}
\begin{frame}[fragile]
%    \frametitle{}

\begin{block}{Find Hierarchy}
 Finding the hierarchy level (number of parents etc.) for each snapshot file (group0 and group1). 
\end{block}
This part may not be a bottleneck part since it only runs for a single snapshot. Another point is, this function probably
is the only one uses the particle coordinates like $x$, $y$, $z$ and $xcen$, $ycen$ etc. in the group data.

Therefore, it implies that we may be able to divide the whole group data into two parts:
\begin{itemize}
 \item Part I is for parent finding in terms of group and particles;
 \item Part II contains other data fields that not used for the parent finding. 
\end{itemize}
Such re-configuration may enhance the parent finding efficiency and memory usage.


\end{frame}

% -------------------------------------------------------------
\begin{frame}[fragile]
%    \frametitle{}

\begin{block}{Find fof parents and find all parents}
 \begin{verbatim}
form arrays to extract ID data from group 1;
loop over group data in group0:
     loop over group ID data in group 1:
          compare the ID data between groups;
          update the group data in group0;
          update the corresponding group data in group1;
     end loop
end loop
 \end{verbatim}
\end{block}
Both of the two functions contains similar structure like above. Currently I suspect this is the bottle-neck part for the program, hence this will be the key part for big data transformation and for optimization. For big data application, this part need to be carefully designed to avoid the communication penalty.

\end{frame}

% -------------------------------------------------------------
\begin{frame}[fragile]
%    \frametitle{}

\begin{block}{Find parents for remaining groups in group0}
 \begin{verbatim}
loop over remaining snapshot data files:
     free the group1 data array;
     initialize and load in the group1 data array;
     find hierarchy for group1;
     find fof parents and find all parent for group; 
     data in group0;
end loop
 \end{verbatim}
\end{block}
this step is to reuse the previous functions so that to searching the remaining unknown parent halos.

\end{frame}

% -------------------------------------------------------------
\begin{frame}[fragile]
%    \frametitle{}

\begin{block}{Remaining functions: check\_fof\_matches and find\_progenitor}
both of the two functions has similar structure with parents finding functions, they are only used for two adjacent snapshots.
\end{block}

\end{frame}

% -------------------------------------------------------------
\section{Suggestions to Improve Code}
\begin{frame}
%    \frametitle{}

\begin{block}{List for Improving:}
 \begin{itemize}
  \item improve the way to use macro so that code is easier for reading;
  \item using high level data structure such as vector etc. to replace the current 
  group array pointer;
  \item using Object-Oriented Programming (OOP) for encapsulation. The final codes 
  will be a collection of classes and the whole HaloParentFinder will be a module. This 
  step is important for future big data application transformation;
  \item improve the group data structure so that to reduce the memory usage;
  \item revise the reading of group data so that loading part/whole snapshot data is 
  possible;
  \item revise the code structure for searching functions.
 \end{itemize}
\end{block}
 
\end{frame}

% -------------------------------------------------------------
\subsection{Use of Macro}
\begin{frame}[fragile]
%    \frametitle{}

a piece of code example in main.c:
 \begin{verbatim}
#ifdef SUBFIND
  my_snprintf(outfname, MAXLEN,"%s/groups_%03d.fofcat", 
       PARAMS.GROUP_DIR,snapshot_number);    
  NFof0 = returnNhalo(outfname);
#endif

#ifndef FOF_ONLY
#define RETURN_ONLY_FOFS 1  
#endif
   
#ifdef SUSSING_TREES
   my_snprintf(outfname,MAXLEN,"%s/%s%05d.z%5.3f.AHF_halos", 
      PARAMS.GROUP_DIR, PARAMS.GROUP_BASE,snapshot_number,REDSHIFT[snapshot_number]);
  NFof0 = returnNhalo_SUSSING(outfname,RETURN_ONLY_FOFS);
#endif

#ifdef BGC2
  my_snprintf(outfname,MAXLEN,"%s/halos_%03d.0.bgc2",  PARAMS.GROUP_DIR, snapshot_number);
  NFof0 = returnNhalo_bgc2(outfname,RETURN_ONLY_FOFS);
#endif  

#ifndef FOF_ONLY  
#undef RETURN_ONLY_FOFS
#endif
 \end{verbatim}
 
\end{frame}

% -------------------------------------------------------------
\begin{frame}[fragile]

How can we express it in another way:
\begin{verbatim}
# debugging code....
if (params.doDebug(BGC2)) {
   .....
} else if (params.doDebug(FOF_ONLY)) {
   .....
}

# calculate the number of halos
# we can also calculate nhalos for different situations
NFof0 = returnNhalo(params,....)
\end{verbatim}

\end{frame}


% -------------------------------------------------------------
\subsection{Data and Code Structure}
\begin{frame}

\begin{block}{Redesign the data structure}
The data structure for group could be divided into two parts:
 \begin{enumerate}
  \item the normal group data(the normal group data is also used in following two programs?);
  \item group data only for parent finding process.
 \end{enumerate}
 The redesign of the code structure enables the loading the group data in more flexible way. 
\end{block}

\end{frame}

% -------------------------------------------------------------
\begin{frame}[fragile]

a general procedure for halo parent finding process:
\begin{verbatim}
# find hierarchy
form normal group data for group0 and group1 to 
   calculate hierarchy (number of parent level);
hold the result into a data structure;

# perform finding parents;
form group0 and group1 data only for parent seaching process;
loading in group and possible particles data from particles file;
load in the hierarchy result;
perform finding parent process;
\end{verbatim}

\end{frame}

% -------------------------------------------------------------
\begin{frame}[fragile]

\begin{block}{questions:}
 \begin{enumerate}
  \item Can we split the large snapshot file into a collection of files?
  \item How the result of searching process is accessed by other programs?
  \item Do the other two programs also use the normal group data?
  \item Does the big data analysis applicable for this situation?
 \end{enumerate}
\end{block}

\end{frame}

% -------------------------------------------------------------
\section{Outlook for further Improvement}
\begin{frame}[fragile]

Previous suggestions are for step 1. Here is the summary of what we can achieve in the 
step 1:
 \begin{enumerate}
  \item memory issue should be solved;
  \begin{itemize}
   \item use of smaller data structure for searching process;
   \item loading data is flexible - we can either load in whole data file, or part of 
   data file
  \end{itemize}
  \item can be used for more snapshots; 
  \item use of high level programming features such as C++/java for programming.
 \end{enumerate}

\end{frame}

% -------------------------------------------------------------
\begin{frame}[fragile]

For step 2, we have two options for consideration:
\begin{block}{big data application transformation}
 \begin{itemize}
 \item  We need to find a situation that big data analysis is applicable here;
 \item  If designed for big data analysis, we need to use Java for implementation.
\end{itemize}
\end{block}

\begin{block}{go parallel}
 We can also use multi-threading such as OpenMP or thread library inside Java to speed up the calculation.
\end{block}

\end{frame}

% -------------------------------------------------------------
\section{Cost Estimation}
\begin{frame}
%    \frametitle{}

An coarse cost estimation for step I implementation:
\begin{itemize}
 \item about 1.5-2 months for code design and code writing;
 \item another 1.5-2 months for debugging and testing.
\end{itemize}


\end{frame}

% -------------------------------------------------------------


\end{document}


