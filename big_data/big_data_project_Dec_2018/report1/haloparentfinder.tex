%!TeX encoding = UTF-8
%!TeX program = xelatex
\documentclass[notheorems, aspectratio=54]{beamer}
% aspectratio: 1610, 149, 54, 43(default), 32

\usepackage{latexsym}
\usepackage{amsmath,amssymb}
\usepackage{mathtools}
\usepackage{color,xcolor}
\usepackage{graphicx}
\usepackage{algorithm}
\usepackage{amsthm}
\usepackage{lmodern} % 解决 font warning
% \usepackage[UTF8]{ctex}
\usepackage{animate} % insert gif

\usepackage{lipsum} % To generate test text 
\usepackage{ulem} % 下划线,波浪线

\usepackage{listings} % display code on slides; don't forget [fragile] option after \begin{frame}
\usepackage{verbatim}
\makeatletter
\def\verbatim@font{\tiny\ttfamily}
\makeatother

% ----------------------------------------------
% tikx
\usepackage{framed}
\usepackage{tikz}
\usepackage{pgf}
\usetikzlibrary{calc,trees,positioning,arrows,chains,shapes.geometric,%
    decorations.pathreplacing,decorations.pathmorphing,shapes,%
    matrix,shapes.symbols}
\pgfmathsetseed{1} % To have predictable results
% Define a background layer, in which the parchment shape is drawn
\pgfdeclarelayer{background}
\pgfsetlayers{background,main}

\definecolor{AmethystPurple}{HTML}{AEAEDF}
% define styles for the normal border and the torn border
\tikzset{
  normal border/.style={AmethystPurple, decorate, 
     decoration={random steps, segment length=2.5cm, amplitude=.7mm}},
  torn border/.style={AmethystPurple, decorate, 
     decoration={random steps, segment length=.5cm, amplitude=1.7mm}}}

% Macro to draw the shape behind the text, when it fits completly in the
% page
\def\parchmentframe#1{
\tikz{
  \node[inner sep=1.5em] (A) {#1};  % Draw the text of the node
  \begin{pgfonlayer}{background}  % Draw the shape behind
  \fill[normal border] 
        (A.south east) -- (A.south west) -- 
        (A.north west) -- (A.north east) -- cycle;
  \end{pgfonlayer}}}

% Macro to draw the shape, when the text will continue in next page
\def\parchmentframetop#1{
\tikz{
  \node[inner sep=2em] (A) {#1};    % Draw the text of the node
  \begin{pgfonlayer}{background}    
  \fill[normal border]              % Draw the ``complete shape'' behind
        (A.south east) -- (A.south west) -- 
        (A.north west) -- (A.north east) -- cycle;
  \fill[torn border]                % Add the torn lower border
        ($(A.south east)-(0,.2)$) -- ($(A.south west)-(0,.2)$) -- 
        ($(A.south west)+(0,.2)$) -- ($(A.south east)+(0,.2)$) -- cycle;
  \end{pgfonlayer}}}

% Macro to draw the shape, when the text continues from previous page
\def\parchmentframebottom#1{
\tikz{
  \node[inner sep=2em] (A) {#1};   % Draw the text of the node
  \begin{pgfonlayer}{background}   
  \fill[normal border]             % Draw the ``complete shape'' behind
        (A.south east) -- (A.south west) -- 
        (A.north west) -- (A.north east) -- cycle;
  \fill[torn border]               % Add the torn upper border
        ($(A.north east)-(0,.2)$) -- ($(A.north west)-(0,.2)$) -- 
        ($(A.north west)+(0,.2)$) -- ($(A.north east)+(0,.2)$) -- cycle;
  \end{pgfonlayer}}}

% Macro to draw the shape, when both the text continues from previous page
% and it will continue in next page
\def\parchmentframemiddle#1{
\tikz{
  \node[inner sep=2em] (A) {#1};   % Draw the text of the node
  \begin{pgfonlayer}{background}   
  \fill[normal border]             % Draw the ``complete shape'' behind
        (A.south east) -- (A.south west) -- 
        (A.north west) -- (A.north east) -- cycle;
  \fill[torn border]               % Add the torn lower border
        ($(A.south east)-(0,.2)$) -- ($(A.south west)-(0,.2)$) -- 
        ($(A.south west)+(0,.2)$) -- ($(A.south east)+(0,.2)$) -- cycle;
  \fill[torn border]               % Add the torn upper border
        ($(A.north east)-(0,.2)$) -- ($(A.north west)-(0,.2)$) -- 
        ($(A.north west)+(0,.2)$) -- ($(A.north east)+(0,.2)$) -- cycle;
  \end{pgfonlayer}}}

% Define the environment which puts the frame
% In this case, the environment also accepts an argument with an optional
% title (which defaults to ``Example'', which is typeset in a box overlaid
% on the top border
\newenvironment{parchment}[1][Example]{%
  \def\FrameCommand{\parchmentframe}%
  \def\FirstFrameCommand{\parchmentframetop}%
  \def\LastFrameCommand{\parchmentframebottom}%
  \def\MidFrameCommand{\parchmentframemiddle}%
  \vskip\baselineskip
  \MakeFramed {\FrameRestore}
  \noindent\tikz\node[inner sep=1ex, draw=black!20,fill=AmethystPurple, 
          anchor=west, overlay] at (0em, 1em) {\sffamily#1};\par}%
{\endMakeFramed}

% ----------------------------------------------

\mode<presentation>{
    \usetheme{Berkeley}
    % Boadilla CambridgeUS
    % default Antibes Berlin Copenhagen
    % Madrid Montpelier Ilmenau Malmoe
    % Berkeley Singapore Warsaw
    \usecolortheme{dolphin}
    % beetle, beaver, orchid, whale, dolphin
    \useoutertheme{infolines}
    % infolines miniframes shadow sidebar smoothbars smoothtree split tree
    \useinnertheme{circles}
    % circles, rectanges, rounded, inmargin
}
% 设置 block 颜色
\setbeamercolor{block title}{bg=AmethystPurple,fg=white}

\newcommand{\reditem}[1]{\setbeamercolor{item}{fg=red}\item #1}

% 缩放公式大小
\newcommand*{\Scale}[2][4]{\scalebox{#1}{\ensuremath{#2}}}

% 解决 font warning
\renewcommand\textbullet{\ensuremath{\bullet}}

% verbatim


% ---------------------------------------------------------------------
% flow chart
\tikzset{
    >=stealth',
    punktchain/.style={
        rectangle, 
        rounded corners, 
        % fill=black!10,
        draw=white, very thick,
        text width=6em,
        minimum height=2em, 
        text centered, 
        on chain
    },
    largepunktchain/.style={
        rectangle,
        rounded corners,
        draw=white, very thick,
        text width=10em,
        minimum height=2em,
        on chain
    },
    line/.style={draw, thick, <-},
    element/.style={
        tape,
        top color=white,
        bottom color=blue!50!black!60!,
        minimum width=6em,
        draw=blue!40!black!90, very thick,
        text width=6em, 
        minimum height=2em, 
        text centered, 
        on chain
    },
    every join/.style={->, thick,shorten >=1pt},
    decoration={brace},
    tuborg/.style={decorate},
    tubnode/.style={midway, right=2pt},
    font={\fontsize{10pt}{12}\selectfont},
}
% ---------------------------------------------------------------------

% code setting
\lstset{
    language=C++,
    basicstyle=\ttfamily\footnotesize,
    keywordstyle=\color{red},
    breaklines=true,
    xleftmargin=2em,
    numbers=left,
    numberstyle=\color[RGB]{222,155,81},
    frame=leftline,
    tabsize=4,
    breakatwhitespace=false,
    showspaces=false,               
    showstringspaces=false,
    showtabs=false,
    morekeywords={Str, Num, List},
}

% ---------------------------------------------------------------------

%% preamble
\title[Halo Parent Finder]{HaloParentFinder}
\subtitle{Big Data Project Transformation}
\author{Fenglai Liu}
\institute[ACCRE]{fenglai@accre.vanderbilt.edu}

% -------------------------------------------------------------

\begin{document}

%% title frame
\begin{frame}
    \titlepage
\end{frame}

% -------------------------------------------------------------
\section{Components}

\subsection{Parameters Reading}
\begin{frame}
%    \frametitle{}

The codes for reading the parameters are in the read\_param.c and read\_param.h. The parameters defines
the location of the groups data(input), and output files directory; the limits of snapshots 
and other debugging options (one data file is a snapshot). The developer use a global variable PARAMS to store 
the parameters information. 

\end{frame}

% -------------------------------------------------------------
\subsection{Halo Groups}
\begin{frame}
%    \frametitle{}

``group'' is the central idea for this program - the structure of group is defined in io.h. Group is a collection of particles(Halo). One data file (snapshot)
may containing may groups.

\begin{block}{reading in the group data}
 \begin{itemize}
  \item first line is the the number of groups data;
  \item the following line is the N particles and the halo ID;
  \item after the lines are the particles data, like  x, y, z, vx, vy, vz etc. are all allocated based on the number of particles.
 \end{itemize}
\end{block}


In the code the group\_data pointer is an array of groups. The group data is initialized in the  the loadgroups
function.


\end{frame}

% -------------------------------------------------------------
\subsection{Functions}
\begin{frame}[fragile]
%    \frametitle{}

\begin{block}{Find Hierarchy}
 Finds the hierarchy level for each snapshot file (group0 and group1). 
\end{block}


\end{frame}

% -------------------------------------------------------------
\subsection{Functions}
\begin{frame}[fragile]
%    \frametitle{}

\begin{block}{Find fof parents and find all parent}
 \begin{verbatim}
loop over group data in group0:
     compare the ID from group1 with the group;
     update this group data in group0;
     update the corresponding group data in group1;
end loop
 \end{verbatim}
\end{block}
Both of the two functions contains similar structure like above. This will be the first key part for big data transformation
and for optimization.
\begin{itemize}
\item  is there a chance to use a small data structure for the computation?
\item  if for big data application, how to do it avoiding the communication penalty?
\end{itemize}

\end{frame}



% -------------------------------------------------------------

\subsection{General Procedure}
\begin{frame}
%    \frametitle{}

\begin{itemize}
 \item Load in parameters; the global parameters are stored in a global struct PARAMS;
 \item Calculate the number of one groups - group0 and group1. Here the number of groups is got from the input parameter file,
 rather than the group data file itself. The first data in the input parameter file seems to be the number of groups;
 \item Allocate space for group0 and group1 and loading in the group data; 
 \item Finds the hierarchy level (this is for a single snapshot);
 \item Find fof parents;
 \item Find all parents;
 \item 
\end{itemize}


\end{frame}





% -------------------------------------------------------------
\section{Bad Design}
\begin{frame}
%    \frametitle{}

\begin{block}{Rigid data structure} 
group data array: the reading of NGroups is from a different place 
 \end{block}


 \begin{block}{Using the global variable} 
  ``readshift'' array as an example:  used under macro ``SUSSING\_TREES''. In improving the program we need to consider
  the matching between snapshot data and redshift data.
 \end{block}


\end{frame}

% -------------------------------------------------------------
\begin{frame}
      
\end{frame}

% -------------------------------------------------------------
\subsection{Design of Scala}
\begin{frame}
  
\end{frame}

% -------------------------------------------------------------
\begin{frame}


\end{frame}


% -------------------------------------------------------------
\section{Features of Scala}
\subsection{Type inference for Scala}
\begin{frame}


\end{frame}


% -------------------------------------------------------------


\end{document}


